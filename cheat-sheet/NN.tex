\documentclass[11pt, twocolumn]{article}
\usepackage{amsmath}
\usepackage{geometry}
 \geometry{ a4paper, total={170mm,257mm}, left=7mm, top=10mm, right=7mm, bottom=10mm}
 
\pagenumbering{gobble}

\newcommand{\argmin}{\operatornamewithlimits{argmin}}
\newcommand{\argmax}{\operatornamewithlimits{argmax}}
\newcommand{\sign}{}

\title{Neural Networks cheat-sheet}
\author{Andrea Jemmett}
\date{\today}

\begin{document}

\maketitle

\section{Perceptron}
Perceptron Convergence Algorithm

\footnotesize
For linearly sep. data; decision boundary: $\sum \mathbf{w_i x_i} + b = 0$
\begin{enumerate}
	\item Variables \& parameters:\\
	$\mathbf{x}(n) = \left[1, x_1(n), \dots, x_m(n)\right]^T$\\
	$\mathbf{w}(n) = \left[b, w_1(n), \dots, w_m(n)\right]^T$\\
	$y(n) = \mbox{net out}$\qquad$d(n) = \mbox{target}$\qquad$\eta = \mbox{learning rate}$
	\item \emph{Initialization} $\mathbf{w}(0) = \mathbf{0}$ then for $n = 1, 2, \dots$ do the following
	\item \emph{Activation} Feed input $\mathbf{x}(n)$ to network
	\item \emph{Compute actual response} as $y(n) = \sign(\mathbf{w}^T(n)\mathbf{x}(n))$
	\item \emph{Adaptation of weight vector} update weights using:\\
	$\mathbf{w}(n+1) = \mathbf{w}(n)+\eta[d(n)-y(n)]\mathbf{x}(n)$\\
	where:\\
	$d(n) = \begin{cases} +1 & \mbox{if }x(n)\mbox{ belongs to class }C_1 \\ -1 & \mbox{if }x(n)\mbox{ belongs to class }C_2 \end{cases}$
\end{enumerate}
\normalsize

\section{Statistic Based Methods}
\begin{enumerate}
	\item \emph{Observation density / class-conditional / likelihood}\\ $P(X=x|C_i) = \frac{\mbox{\# x samples}}{\mbox{\# of samples in }C_i}$
	\item \emph{Prior} $P(C_i) = \frac{\mbox{\# samples in }C_i}{\mbox{\# all samples}}$
	\item \emph{Posterior} $P(C_i|X=x) = \frac{\mbox{likelihood x prior}}{\mbox{evidence}}$
	\item \emph{Evidence} $P(X=x)$ is normalization / scaling factor
\end{enumerate}
$$\mathbf{w_{MAP}} = \argmax_{\mathbf{w}} \pi(\mathbf{w}|d,\mathbf{x})$$

\section{Linear Models}
\footnotesize
\textbf{Gradient Descent Algorithm}
\begin{enumerate}
	\item Start from arbitrary point $\mathbf{w}(0)$
	\item find a direction by means of a gradient: $\nabla\xi = [\frac{\partial f}{\partial w_1}, \dots, \frac{\partial f}{\partial w_n}]$
	\item make a small step in that direction: $\Delta\mathbf{w} = -\eta\nabla\xi$
	\item repeat the whole process
\end{enumerate}
\normalsize
\textbf{ADALINE} uses an identity activation (continuous error measure) function and update rule is
$$\Delta\mathbf{w} = + \eta\mathbf{x}(d-y)$$
\textbf{Linear regression} uses sigmoid activation function and delta rule is
$$\Delta\mathbf{w} = + \eta(d-\varphi(net))\varphi'(net)\mathbf{x}$$
\textbf{Cover's Theorem} "A complex pattern-classification problem, cast in a high dimensional space nonlinearly, is more likely to be linearly separable than in a low-dimensional space, provided that the space is not densely populated."

\section{Multi-Layer Perceptrons}
Considering NN for XOR:
\begin{itemize}
	\item Network output $y = \varphi(net) = \varphi(y_1u_1 + y_2u_2) = \varphi(u_1\varphi(net_1) + u_2\varphi(net_2))$
	\item Network error $e = \frac{1}{2}(d-y)^2$
\end{itemize}
Generalized Backprop delta rule
$$\Delta w_{ji} = \eta \delta_j y_i$$
$$
\delta_j = \begin{cases} \varphi '(v_j)(d-y_j) & \mbox{if }j\mbox{ is output node} \\
 \varphi '(v_j)\sum_{k} \delta_k w_{kj} & \mbox{ if }j\mbox{ is hidden node} \end{cases}
$$

\section{Radial-Basis Function nets}
Main idea: build local model of reference points and combine them.
\begin{itemize}
	\item \emph{Hidden layer} returns closeness from reference points
	\item \emph{Output layer} standard linear regression (like ADALINE)
	\item \emph{Closeness} is a radial function of the Euclidean distance:\\
	\centerline{$\phi(||x-t_i||)$\qquad$\phi(r) = exp(-\frac{r^2}{2\sigma^2})$}\\
	\centerline{$\phi(r) = (r^2 + \sigma^2)^{-\alpha},\quad\alpha>0$\qquad$\phi(r) = r^2\ln(r)$}
\end{itemize}
Training:
\begin{enumerate}
	\item Learn centres using K-means (unsupervised)
	\item Learn weights from hidden to output using LMS (supervised)
\end{enumerate}

\section{Support Vector Machines}
Main idea: \textbf{maximize margin around decision hyperplane}; decision function is specified
by a subset of training samples: the support vectors.
\begin{align}
	\mathbf{w_o^Tx_i}+b_o\geq+1 && \mbox{when }d_i=+1 \\ 
	\mathbf{w_o^Tx_i}+b_o\leq-1 && \mbox{when }d_i=-1
\end{align}
Maximizing the margin of separation $\rho$ is equivalent to minimize the Euclidean norm of the weight vector \textbf{w}
$$\rho = \frac{2}{||\mathbf{w_o}||}$$
\textbf{Problem.} Find values of \textbf{w} that minimize
$$\phi(\mathbf{w}) = \frac{1}{2}\mathbf{w^Tw}$$
given constraints
$$d_i(\mathbf{w^Tx_i+b})\geq1\qquad\mbox{for }i=1,2,\dots,N$$
\textbf{Lagrangian function} (linearly separable) ($\alpha_i>0$ for support vectors)
$$J(\mathbf{w},b,\alpha)=\frac{1}{2}\mathbf{w^Tw}-\sum_{i=1}^{N}\alpha_i[d_i(\mathbf{w^Tw}+b)-1]$$
Solution
$$Q(\alpha)=\sum_{i=1}^N\alpha_i-\frac{1}{2}\sum_{i=1}^N\sum_{j=0}^N\alpha_i\alpha_jd_id_j\mathbf{x_i^Tx_j}$$
By setting partial derivatives to zero we obtain
$$\mathbf{w}=\sum_{i=1}^n \alpha_iy_i\mathbf{x_i}$$
Solution Non linear with kernel function $K(x_i,x_j)$
$$Q(\alpha)=\sum_{i=1}^N\alpha_i-\frac{1}{2}\sum_{i=1}^N\sum_{j=0}^N\alpha_i\alpha_jd_id_j\phi(\mathbf{x_i^T})\phi(\mathbf{x_j})$$
Possible \textbf{kernel functions}
\begin{itemize}
	\item polynomial $K(x,y)=(xy+1)^p$
	\item RBF gaussian $K(x,y)=\exp(-\frac{||x-y||^2}{2\sigma^2})$
	\item sigmoid $K(x,y)=\tanh(kxy-\delta)$
\end{itemize}

\section{Principal Component Analysis}
Main idea: \textbf{discover significant patterns or features in input data through use of unlabelled examples}.
\begin{description}
	\item \emph{Statistical method} \\
		\footnotesize
		\textbf{Step 1} start with a dataset and subtract mean\\
		\textbf{Step 2} calculate covariance matrix\\
		\textbf{Step 3} calculate the eigenvectors and eigenvalues; the eigenvector with the highest eigenvalue is \emph{the} principal component
		\textbf{Step 4} select how much to reduce the dimension of the dataset by selecting the $p$ highest principal components
		\normalsize
	\item \emph{Self-organization} \\
		\footnotesize
		\textbf{Self-amplification} Hebb's postulate: If two neurons on either side of a connection are activated simultaneously (i.e. synchronously), then the strength of that connection is selectively increased. In the case of asynchronous activation the connection strength is decreased \\
		\textbf{Competition} neurons compete with each other in a winner takes all fashion \\
		\textbf{Cooperation} modification in synaptic weights and neurons tend to cooperate \\
		\textbf{Structural information} redundant information is acquired in the form of knowledge \\
		\textbf{Hebbian learning rule}
		\normalsize
		$$\Delta w_{ji}=\eta(y_ix_i-y_i\sum_{k=1}^jw_{ki}y_k)$$
\end{description}

\section{Self-Organizing Maps}
Goal: Transform an incoming signal pattern of arbitrary dimension into a one- or two-dimensional discrete map, and perform this translation adaptively in a topologically ordered fashion \\
\textbf{Three processes:}
\begin{description}
	\item[1. Competition]: find the winning neuron: $i(\mathbf{x})=\argmin_j \Vert \mathbf{x}-\mathbf{w_j}\Vert$
	\item[2. Cooperation]: determine neighbourhood function: $h_{j,i} = \exp(-\frac{d_{j,i}^2}{2\sigma^2})$ \quad where $d_{j,i}$ is the lateral distance from winning neuron
	\item[3. Adaptation]: adapt weights with:\\
	$\mathbf{w_j}(n+1) = \mathbf{w_j}(n)+\eta(n)h_{j,i}(n)(\mathbf{x}(n)-\mathbf{w_j}(n))$
\end{description}
\textbf{Two phases:}
\begin{enumerate}
	\item Ordering phase: topological ordering of the weight vectors; can take 1000 iterations with $\eta=[0.1, 0.01]$
	\item Convergence phase: fine tune the map and provide accurate statistical quantification (small $\eta$)
\end{enumerate}
\textbf{Contextual maps}
Visualize SOM in a different way. \emph{Feature map} see which neurons are excited when an unseen test pattern is presented. \emph{Semantic map} creates a form of clustering or unsupervised categorization

\end{document}
